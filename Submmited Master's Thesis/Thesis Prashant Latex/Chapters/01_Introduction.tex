\chapter{Introduction} 
Gas sensors are essential instruments in many fields. These sensors are mainly used for safety, and they can identify dangerous gases such as carbon monoxide, hydrogen sulfide, methane, or any volatile organic compounds(VOCs).\cite{singh2024metal} They are vital for avoiding mishaps and protecting people's health in working environments. Identifying contaminants in indoor and outdoor environments is essential to environmental monitoring\cite{shitashima2010evolution} because it helps reduce pollution and protect public health, even early disease detection. For the food \cite{acock1995simple} and pharmaceutical industries\cite{severinghaus1986history}, precise control over gas concentrations is essential in industrial processes provided by gas sensors to preserve product quality and safety. Gas sensors are used in medical devices to measure carbon dioxide and oxygen levels, which are vital for patient care, as well as exhaust emissions in automotive applications. By warning residents and first responders about possible fire threats, gas sensors help fire detection systems improve safety. To put it simply, gas sensors are invaluable resources that support environmental preservation, safety, and quality assurance in a variety of sectors.

\section{Gas Sensor Array}
When using a single gas sensor for detection, there may be many reasons why the results could be false or inaccurate. As a result, sensor arrays need to be used. The ability of an array to improve accuracy by cross-referencing data from various sensors is one of its significant advantages. Different gases frequently create similar reactions on a single sensor, which could result in misunderstandings or inaccurate results. On the other hand, an array can more accurately distinguish between different gases by combining sensors with different selectivity.
Additionally, sensor arrays' redundancy guarantees fault tolerance, which is essential in critical applications requiring ongoing monitoring. Even though a single gas sensor can give helpful information, an array of sensors can provide advantages like fault tolerance, enhanced accuracy, selective detection, broader dynamic range, continuous calibration, and better spatial coverage. An array of sensors is essential for critical applications where accuracy and dependability are crucial.

The complexity of data from an array can be too complex. For this work, resistance, the concentration of gas, temperature, sensor element and the number of sensors are the parameters collected, so with these features, the data becomes too complex to visualise. With this complexity, predicting the target/result using the usual linear methods is difficult. So, some non-linear method in Machine Learning is required to fit the data and predict the gas based on given new parameters. With an array of gas sensors, it becomes handy to check the result, as individual sensors can have a false result. For that, supervised Machine learning algorithms such as KNN, Decision Tree, Random Forest, and Neural Networks were used to classify the data for four different gases.

In this project, we have analysed data collected from three sensors exposed to four different gases and explored classification. Also designed hardware where four to six sensors can be accommodated, and data can be collected simultaneously.

\section{Classifiers in ML}
Classifiers in machine learning are algorithms that categorise or classify input data according to its features. These algorithms apply labelled training data to identify patterns, which can be applied to predict the labels of new data points. A key component of supervised learning is the classifier, in which the computer gains knowledge from input-output relationships given during training.

In machine learning, classifiers are algorithms that assign labels or categories to input data based on their features. These algorithms learn patterns from labelled training data and then use this knowledge to predict the labels of new, unseen data points. Classifiers are a fundamental part of supervised learning, where the algorithm learns from input-output pairs provided during training.\\
\textbf{Binary Classifiers}\\
These classifiers classify inputs into one of two classes or categories, such as "spam" or "not spam," "positive" or "negative," etc.\\
\textbf{Multiclass Classifiers}\\
These classifiers classify inputs into one of multiple classes or categories. For example, classifying images of animals into "dog," "cat," "bird," etc.\\
\textbf{Linear Classifiers}\\
These classifiers separate classes using a linear decision boundary. Examples: logistic regression and linear support vector machines (SVM).\\
\textbf{Non-Linear Classifiers}\\
These classifiers can capture non-linear relationships between features and labels. Examples: decision trees, random forests, k-nearest neighbours (KNN), and neural networks.\\
\textbf{Probabilistic Classifiers}\\
These classifiers predict the class label and provide the probability of a data point belonging to each class. Examples include logistic regression and Naive Bayes classifiers.\\
\textbf{Deep Learning Classifiers}\\
These classifiers are based on deep neural networks, which can automatically learn hierarchical representations of data. They are particularly effective for tasks such as image classification, natural language processing, and speech recognition.

Our goal is to predict the gas using four parameters. We also know there are only four types of gas in our dataset, so we can use one of the classifiers, such as Multiclass Classifiers or Non-Linear Classifiers. Since our dataset performs better with non-linear classifiers, our approach used non-linear classifiers, including decision trees, random forests, K-nearest neighbours, and neural networks.

\subsection{Non-Linear Classifier}
Non-linear Classifier can capture Complex relationships between input features and target labels; here, we prefer the Non-Linear approach because linear classifiers assume that the decision boundary separating different classes is a straight line or a hyperplane in higher dimensions; non-linear classifiers are capable of capturing more intricate decision boundaries even in an irregular shape.

\subsubsection{k-Nearest Neighbors (KNN)}KNN is a simple instance-based learning algorithm where the prediction for a new data point is based on the labels of its k nearest neighbours in the feature space. KNN is non-parametric and can learn complex decision boundaries directly from the training data.
\subsubsection{Decision tree}Decision trees recursively partition the feature space into smaller regions by making decisions based on feature values at each node. Each decision leads to a split, creating branches that eventually terminate in leaf nodes representing the predicted class labels.
\subsubsection{Random Forest}Random forests are ensembles of decision trees where each tree is trained on a random subset of the training data and a random subset of features. The final prediction is made by aggregating the predictions of individual trees, often by a majority vote.
\subsubsection{Neural Networks}Neural networks, particularly deep neural networks, are highly flexible non-linear classifiers capable of learning complex mappings between inputs and outputs through multiple layers of interconnected neurons. It has achieved state-of-the-art performance in various machine learning tasks, including image and speech recognition, natural language processing, and many others. We worked and spent more time on the neural Network Approach to Our Problem.