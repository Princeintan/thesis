%~~~~~~~~~~~~~~~~~~~~~~~~~~~~~~~~~~~~~~~~~~~~~~~~~~~~~~~~~~~~~~~~~~~~~~~~~~~~~~~~~~~~~~~~~%
%	                                     ABSTRACT   	                                  %
%~~~~~~~~~~~~~~~~~~~~~~~~~~~~~~~~~~~~~~~~~~~~~~~~~~~~~~~~~~~~~~~~~~~~~~~~~~~~~~~~~~~~~~~~~%
\vspace{6pt}


\begin{flushleft}
    \setlength{\parskip}{0pt}
        %\bigskip
    {\centering{{\Large{\bf{ABSTRACT}}}} \par}
    \bigskip
    \vspace{6pt}
\end{flushleft} % This section is not essential for the abstract
The execution of experiments involves and creates a large amount of data that needs to be acquired, organized and assessed to find a meaningful outcome of the experiments. The main aim of the project is to understand and create a Machine-learning model to identify the gases present in a system based on information provided by the sensors. To begin with we started to learn Principal Component Analysis (PCA), a well-known technique in data analytics. Various methods and techniques are studied and have been implemented in the well-known data sets. A python-based library is generated to perform PCA. It was compared with the commercially available SKlearn library.  Apart from this, to collect thermal expansion data of the samples, a device was built which can measure expansion in a temperature range varying from room temperature to 900 K. The device needs to be optimized for collecting reliable data which is the work under progress.To understand the mathematics behind some useful Machine Learning Algorithms for classification problems and to apply these algorithm to a given set of data generated by Gas Sensors and identify the gases based upon the parameters given in dataset. for that classification algorithms such as K-Nearest Neighbour, Decision Tree, Random Forest and specially neural networks were used. before classification, PCA(Principal Component Analysis) was used in attempt to reduce the number of features/parameters in the dataset. 

\vspace{3cm} 

%{\Large \textbf{Keywords: }}\par{\Large [Insert Keywords]}
\vfill
%{\centering \normalsize Indian Institute of Science Education and Research Thiruvananthapuram \\}
%{\centering \normalsize Thiruvananthapuram - 695 551 \par}
\clearpage